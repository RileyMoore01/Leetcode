\documentclass{article}
\usepackage[utf8]{inputenc}
\usepackage[linguistics]{forest}

\title{CS3383 Homework 2}
\author{Riley Moore}
\date{February 2022}

\begin{document}

\maketitle

\section{NA}

\section{}
(N, T) where\\
N = [a, b, c1, c2, d, e] \\    
T = [(a,b), (b,c1), (b,c2), (a,d), (d,e)]

\section{}
Write a mathematical definition

A = B? NO

A = C? YES

A = D? NO

\section{}


\section{}
\subsection{}
Name: function \\
Parameters: sets A & B
\subsection{}
Names & Arguments = [sets, memberships]
\subsection{}
Logic Symbols: Quantifier = [for every, implication]
\subsection{}
Important Variables: [f, A, B, a, b]
\subsection{}
\begin{center}
\begin{forest}
[$if$
    [$\forall (a \in A)$]
    [$(a, b) \in (f)$]
]
\end{forest}
\end{center}

\section{}
For the set A, we have a set of natural numbers N = 0,1,2,3,...\\
Using function f(n) we can prove:\\
n=0,f(0)=1,n=1,f(1)=3,n=2,f(2)=5...\\
Therefore, set A is countably infinite

\section{}
From the function give, we can deduce:\\
f(1)=2$^1=2$, f(2)=2$^2=4$, f(3)=2$^3=8$, f(4)=2$^4=16$\\
\begin{center}
\begin{forest}
[1
    [2
        [3
            [4
            ]
        ]
    ]
]
\end{forest}
\end{center}

----------------------------------------------------------------------------------------------

\begin{center}
\begin{forest}
[2
    [4
        [8
            [16
            ]
        ]
    ]
]
\end{forest}
\end{center}
How to read the definition: 
a set of D has an element x in the set A and x is not an element in f(x)\\
Therefore we can see that D = 1 and 3


\end{document}
